% Copyright 2007 by Till Tantau
%
% This file may be distributed and/or modified
%
% 1. under the LaTeX Project Public License and/or
% 2. under the GNU Public License.
%
% See the file doc/generic/pgf/licenses/LICENSE for more details.

\ProvidesFileRCS $Header: /cvsroot/pgf/pgf/generic/pgf/basiclayer/pgfcoretransparency.code.tex,v 1.5 2013/09/30 18:30:15 tantau Exp $


% Sets the opacity of stroking operations.
% 
% #1 = opacity, where 1 means fully opaque and 0 means fully
% transparent. 
%
% Example:
%
% \pgfsetstrokeopacity{0.5}

\def\pgfsetstrokeopacity#1{%
  \pgfmathparse{#1}%
  \expandafter\pgfsys@stroke@opacity\expandafter{\pgfmathresult}}


% Sets the opacity of stroking operations.
% 
% #1 = opacity, where 1 means fully opaque and 0 means fully
% transparent. 
%
% Example:
%
% \pgfsetfillopacity{0.5}

\def\pgfsetfillopacity#1{%
  \pgfmathparse{#1}%
  \expandafter\pgfsys@fill@opacity\expandafter{\pgfmathresult}}



% Sets the current blend mode. See Section 7.2.4 of the PDF
% Specification 1.7 for an introduction to blend modes in general. 
% 
% #1 = One of the following modes, see also section 7.2.4 of the
% PDF Specification 1.7. (The PGF names, which are all lowercase with
% spaces as usual, must be mapped to the PDF or SVG by the drivers.)
%
% normal
% multiply
% screen
% overlay
% darken
% lighten
% color dodge
% color burn
% hard light
% soft light
% difference
% exclusion
% saturation
% color
% hue
% luminosity
%
% Example:
%
% \pgfsetblendmode{multiply}

\def\pgfsetblendmode#1{\pgfsys@blend@mode{#1}}


% Declares a new fading based on the contents of a box.
%
% #1 = name of the fading
% #2 = content of a TeX box that specifies the fading
%
% Description:
%
% This command declares a new fading based on the contents of
% the TeX box. The mask can later be used to mask other graphics.
%
% The box is used to determine where the mask is opaque: At the
% beginning, the mask is completely transparent. Whereever the box
% contains something (like text or a picture or whatever), the
% luminosity of the text/graphic is used to determine how opaque the
% point will be. The greater the luminosity, the more opaque. The net
% effect is that white corresponds to "completely opaque" and black to
% "completely transparent". This is rather counterintuitive. Because
% of this, the special color "pgftransparent" is defined, which is the
% same as black. By setting the color to, say,
% \color{pgftransparent!10} you get something that is 10% transparent
% and \color{pgftransparent!90} gives you 90% transparency.
%
% By default, the drawing color is set to opaque.
%
% Example:
%
% \pgfdeclarefading{mymask}{
%  \fontsize{200}{200}\bfseries
%  Ti\emph{k}Z
% }
% \pgfdeclarefading{myothermask}{
%  \tikz \shade [left color=transparent,right color=transparent!0]
%    (0,0) rectangle (5,5);
% }

\def\pgfdeclarefading#1#2{%
  \setbox\pgfutil@tempboxa=\hbox{{\pgfsys@beginscope\pgfutil@color{white}\ignorespaces#2\pgfsys@endscope}}%
  \pgfsys@fadingfrombox{#1}\pgfutil@tempboxa%
}

\pgfutil@colorlet{pgftransparent}{black}


% Use a previously declared fading.
%
% #1 = name of the fading
% #2 = transformation code
%
% Description:
%
% The command should be used *inside* a pgfpicture to install a
% fading. The fading is a part of the graphic state. (This command 
% works much like \pgfsetstrokeopacity).
%
% The mask will be centered on the origin. The transformation code #2 is
% applied to the mask (but not the contents of the actual picture).

\def\pgfsetfading#1#2{%
  {%
    \pgftransformreset%
    #2%
    \pgfgettransform{\pgftrans@temp}%
    \edef\pgf@marshal{\noexpand\pgfsys@usefading{#1}\pgftrans@temp}%
    \pgf@marshal%
  }%
}



% Use a previously declared fading scaled for the current path.
%
% #1 - an fading
% #2 - extra transformations
%
% Description:
%
% This command works a bit like \pgfshadepath. "Works a bit like"
% means the following: The masks's original size should
% completely cover the area between (0,0) and (100bp,100bp). The
% mask is then rescaled so that it completely covers the path. Then
% the additional transformation #2 is applied. Finally, the mask is
% made part of the graphic state (as with \pgfsetfading). The path
% used for computation purposes only -- it is not drawn or otherwise
% used.  
%
% If no transformations occur, the lower left corner of the path will
% lie on (25bp, 25bp), the upper right corner on (75bp, 75bp).
%
% If the path was empty, the fading will be used at its natural size,
% centered on the origin.
%
% Example:
%
% \pgfdeclarefading{fade right}
% {\tikz[left color=transparent!0,right color=transparent!100]
%    (0,0) rectangle (100bp,100bp);}
%
% \pgfpathmoveto{\pgforigin}
% \pgfpathlineto{\pgfxy(1,0)}
% \pgfpathlineto{\pgfxy(1,1)}
% \pgfsetfadingforcurrentpath{fade right}{}
% \pgfusepath{fill}

\def\pgfsetfadingforcurrentpath#1#2{%
  \ifdim\pgf@pathminx=16000pt%
    % Path was empty -> use natural size
    \pgfsetfading{#1}{#2}%
  \else%
    \pgfsetfading{#1}{%
      % Calculate center:
      \pgf@xb=.5\pgf@pathmaxx%
      \advance\pgf@xb by.5\pgf@pathminx%
      \pgf@yb=.5\pgf@pathmaxy%
      \advance\pgf@yb by.5\pgf@pathminy%
      % Calculate scaling:
      \pgf@xc=\pgf@pathmaxx%
      \advance\pgf@xc by-\pgf@pathminx%
      \ifdim\pgf@xc<0.1pt\pgf@xc=0.1pt\fi%
      \pgf@yc=\pgf@pathmaxy%
      \advance\pgf@yc by-\pgf@pathminy%
      \ifdim\pgf@yc<0.1pt\pgf@yc=0.1pt\fi%
      \pgf@xc=.02\pgf@xc%
      \pgf@yc=.02\pgf@yc%
      % Compute new transformation matrix:
      \pgftransformcm{1}{0}{0}{1}{\pgfqpoint{\pgf@xb}{\pgf@yb}}%
      \pgftransformcm{\pgf@sys@tonumber{\pgf@xc}}{0}{0}{\pgf@sys@tonumber{\pgf@yc}}{\pgfpointorigin}%
      #2%
    }%
  \fi%
}




% Use a previously declared fading scaled for the current path.
%
% #1 - an fading
% #2 - extra transformations
%
% Description:
%
% This command works like \pgfsetfadingforcurrentpath, only it takes
% the current line width into account: Before the path is considered,
% it is enlarged by half the line width in all directions.

\def\pgfsetfadingforcurrentpathstroked#1#2{%
  \ifdim\pgf@pathminx=16000pt%
    % Path was empty -> use natural size
    \pgfsetfading{#1}{#2}%
  \else%
    \pgfsetfading{#1}{%
      % Calculate center:
      \pgf@xb=.5\pgf@pathmaxx%
      \advance\pgf@xb by.5\pgf@pathminx%
      \pgf@yb=.5\pgf@pathmaxy%
      \advance\pgf@yb by.5\pgf@pathminy%
      % Calculate scaling:
      \pgf@xc=\pgf@pathmaxx%
      \advance\pgf@xc by-\pgf@pathminx%
      \advance\pgf@xc by\pgflinewidth%
      \ifdim\pgf@xc<0.1pt\pgf@xc=0.1pt\fi%
      \pgf@yc=\pgf@pathmaxy%
      \advance\pgf@yc by-\pgf@pathminy%
      \advance\pgf@yc by\pgflinewidth%
      \ifdim\pgf@yc<0.1pt\pgf@yc=0.1pt\fi%
      \pgf@xc=.02\pgf@xc%
      \pgf@yc=.02\pgf@yc%
      % Compute new transformation matrix:
      \pgftransformcm{1}{0}{0}{1}{\pgfqpoint{\pgf@xb}{\pgf@yb}}%
      \pgftransformcm{\pgf@sys@tonumber{\pgf@xc}}{0}{0}{\pgf@sys@tonumber{\pgf@yc}}{\pgfpointorigin}%
      #2%
    }%
  \fi%
}



% Group things in a a transparency group
%
% Description:
%
% This takes groups pgf commands inside a pgfpicture in a transparency
% group. This means that any transparency settings apply to commands a
% whole. For instance, if a box contains two overlapping black 
% circles and you draw them normally with 50% transparency, then the
% overlap will be darker than the rest. By comparison, if the circles
% are part of a transparency group, the overlap will get the same
% color as the rest.
%
% 

\def\pgftransparencygroup{%
  \pgfutil@ifnextchar[{\pgftransparencygroup@}{\pgftransparencygroup@[]}%
}
\def\pgftransparencygroup@[#1]{%
  \begingroup%
    \pgfkeys{/pgf/transparency group options/.cd,%
      isolated=false,
      knockout=false,
      #1}%
    \setbox\pgfutil@tempboxa=\hbox\bgroup%
}
\pgfkeys{
  /pgf/transparency group options/.cd,
  isolated/.is if=pgfsys@transparency@group@isolated,
  knockout/.is if=pgfsys@transparency@group@knockout
}
  
\def\endpgftransparencygroup{%
    \egroup%
    % Now compute the correct position. This is a bit tricky...
    \ifdim\pgf@picminx=16000pt%
      % Nothing needs to be done
    \else%
      % Safety margin to avoid clipping:
      \advance\pgf@picminx by-1cm%
      \advance\pgf@picmaxx by1cm%
      \advance\pgf@picminy by-1cm%
      \advance\pgf@picmaxy by1cm%
      % Ok, set the box size to current picture size.
      \setbox\pgfutil@tempboxa=\hbox{%
        \vrule width0pt height\pgf@picmaxy depth-\pgf@picminy%
        \hskip-\pgf@picminx%
        \box\pgfutil@tempboxa%
        \hskip\pgf@picmaxx%
      }%
      \pgfsys@transparencygroupfrombox\pgfutil@tempboxa%
      \wd\pgfutil@tempboxa=0pt%
      \ht\pgfutil@tempboxa=0pt%
      \dp\pgfutil@tempboxa=0pt%
      \hskip\pgf@picminx\box\pgfutil@tempboxa\hskip-\pgf@picminx%
    \fi%
  \endgroup%
}

\let\startpgftransparencygroup=\pgftransparencygroup
\let\stoppgftransparencygroup=\endpgftransparencygroup

\endinput
