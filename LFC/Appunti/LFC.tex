\documentclass[a4paper,10pt,oneside,openany]{book}
\usepackage[utf8]{inputenc}
\usepackage{amsmath}
\usepackage{cancel}
\usepackage{amsfonts}
\usepackage{amssymb}
\usepackage{xcolor}
\usepackage{mathrsfs}
\usepackage{amsfonts}
\usepackage{amsmath}
\usepackage{mathtools}
\usepackage{centernot}
\usepackage{graphicx}
\usepackage{setspace}
\usepackage{geometry}
\usepackage{tcolorbox}
\usepackage{hyperref}
\usepackage{stackengine}
%%%%%%%%%%%%%%%%%%%%%%%%%%%%%%%%%%%%%%%%%%%%%%%%%%%%%%%%%%%%%%%%%%%%%%%%%%%%%%%%%%%%%%%%%%%%%%%%%%
\usepackage{multicol}

%%%%%%%%%%%%%%%%%%%%%%%%%%%%%%%%%%%%%%%%%%%%%%%%%%%%%%%%%%%%%%%%%%%%%%%%%%%%%%%%%%%%%%%%%%%%%%%%%%
\usepackage{listings}
\usepackage{color}
\usepackage{qtree}

\definecolor{dkgreen}{rgb}{0,0.6,0}
\definecolor{gray}{rgb}{0.5,0.5,0.5}
\definecolor{mauve}{rgb}{0.58,0,0.82}

\lstset{frame=tb,
  language=C++,
  aboveskip=3mm,
  belowskip=3mm,
  showstringspaces=false,
  columns=flexible,
  basicstyle={\small\ttfamily},
 %numbers=left,
  numbers=none,
  numberstyle=\tiny\color{gray},
  keywordstyle=\color{blue},
  commentstyle=\color{dkgreen},
  stringstyle=\color{mauve},
  breaklines=true,
  breakatwhitespace=true,
  tabsize=3,
  %xleftmargin=0.5cm
}
%%%%%%%%%%%%%%%%%%%%%%%%%%%%%%%%%%%%%%%%%%%%%%%%%%%%%%%%%%
\usepackage{graphicx}
\graphicspath{ {images/} }

\definecolor{grigio}{RGB}{179,179,179}
%\theoremstyle{plain}
\newtheorem{thm}{Teorema}[section]
\newtheorem{deff}{Definizione}[section]
\newtheorem{dimm}{Dimostrazione}[section]

\geometry{a4paper,top=1.5cm,bottom=1.5cm,left=2.5cm,right=2.5cm,%
heightrounded,bindingoffset=5mm}

\newcommand{\tb}[1]{\textbf{#1}}
\newcommand{\e}[1]{\textit{#1}}
\newcommand{\DEF}[1]{\textcolor{blue}{DEF:}}
\newcommand{\om}[0]{$\Omega$}
\newcommand{\al}[0]{$\mathcal{A}$}
\newcommand{\zero}[0]{$\emptyset$}
\newcommand{\ap}[0]{$\in$}
\newcommand{\im}[0]{$\implies$}

\setcounter{tocdepth}{1} %il grado di profondita indice, 1 solo chapter, 2 section ...

\author{Edoardo Lenzi}
\title{LFC (Linguaggi Formali e Compilatori) - Note del Corso}
\begin{document}
\maketitle
\tableofcontents
%\chapter{Introduzione}
Un compilatore \'e un programma che legge un \textbf{linguaggio source} e lo traduce in un \texttt{equilvalente} \textbf{linguaggio di programmazione
target}. 
Solitamente il compilatore compila in \texttt{assembly} e poi un \textbf{assembler} produce codice macchina.
Se il target language e un programma eseguibile pu\'o processare input e produrre output.

Un \textbf{interprete} \'e un altro tipo di language processor, invece di tradurre il linguaggio lo esegue direttamente
quindi piglia sia il source program che gli input e processa l'output

Infine il \textbf{preprocessore} risolve le macro nel sorgente codificandole in linguaggio nativo (espandendole) prima di compilare.

\begin{quote}
Solitamente il compilato va pi\'u veloce mentre l`interprete ti da diagnosi piu accurate dato che esegue il codice.
Nel caso di Java compilo il sorgente in linguaggio intermedio \texttt{bytecode} che poi interpreto sulla JVM.
\end{quote}

Il \textbf{linker} \lq\lq linka\rq\rq\ assieme moduli e librerie dove ho riferimenti ad altri file (risolve gli indirizzi).
Il \textbf{loader} invece fa il merge in memoria per l'esecuzione.

%img4

\section{Front-End of the Compiler}
La \textbf{parte analitica} del processo di compilazione spacca la sorgente in parti costituenti e impone su di esse una struttura 
grammaticale (stile dtd); sfrutta questa struttura per creare una rappresentazione intermedia.
Se non passa la validazione grammaticale mi tira errori. Il sorgente viene storicizzato in
una struttura dati chiamata \textbf{symbol table}. 

\section{Back-End of the Compiler}
La \textbf{parte di sintesi} invece traduce il sorgente guardando la rappresentazione intermedia e la symbol table;
le parti di analisi e sintesi sono chiamate anche \textbf{front-end of the copiler} mentre le restanti \textbf{back-end}.

\section{Lexical analysis}
Fa uno scan e raggruppa le parole in \textbf{lexems}, per ogni lexem genera un \textbf{token} della forma

\begin{center}
    \texttt{(token name, attribute value)} 
\end{center}

Il \texttt{token name} \'e un simbolo astratto usato nella syntax analysis
mentre il \texttt{value} \'e un puntatore alla symbol table entry.\\[5pt]

\textbf{ie)}
\begin{quote}
    \texttt{position = initial + rate * 60} diventa \texttt{(id, 1) (=) (id, 2) (+) (id, 3) (*) (60)}\\
    gli operatori matematici sono simboli astratti che non hanno attribute value (?non sono referenziati nella symbol table?).
\end{quote}

%img7

\section{Session syntax analyzer} 
\'E un parsing, con i token crea una \textbf{rappresentazione ad albero (syntax tree)} nel quale 
il nodo \'e un operatore e i figli gli operandi. 

gli operatori devono avere priorit\'a per costruire l'albero, la struttura grammaticale serve anche a 
definire le priorit\'a degli operatori.

\section{Semantic analyzer} 
Piglia il \texttt{syntax tree} e guarda se \'e semanticamente consistente con la definizione del linguaggio.
(ie \texttt{type checking}). Il linguaggio puo ammettere cast impliciti chiamati \textbf{coercizioni} o tirare cogne.

\lq\lq \textit{intofloat}\rq\rq \'e una coercizione dell'intero 60 in float dato che gli altri operandi sono float. 

\section{Intermediate code generation}
Nel processo di compilazione posso avere varie rappresentazioni intermedie come alberi etc..
Dopo semantic analysis solitmente creo una codice basso livello, machine-like, \textit{\lq\lq easy to 
produce and esay to translate int target machine code\rq\rq}. Nella figura ho un tree address code
ricavato dal syntax tree. 

In un tree address code a destra ho al massimo un operatore (assembly like), e le operazioni sono in ordine.

Devo avere variabiline intermedie
\section{Code generation}
Segue la fase opzionale di \textbf{code optimization}, prende la rappresentazione inermedia e la mappa in un target language.
Le istruzioni intermedie vengono tradote in istruzioni macchina (presumibilmente). Devo capire come mappare variabili su registri

Nella symbol table devo storicizzare tutti gli attributi di un variable name.
 
Solitamente posso agglomerare le fasi di analisi in front end pass e le altre in back end pass.







%\chapter{Introduction}
\begin{tcolorbox}\begin{center}
Un sistema \'e un entit\'a artificiale o fisica che evolve nel tempo. Spesso conviene definire un sistema come la relazione fra segnali in input e in output.
\end{center}\end{tcolorbox}
Un \textbf{segnale} \'e una funzione che descrive il sistema nel suo evolvere di quantit\'a col passare del tempo. 

\begin{tcolorbox}\begin{center}
    Segnale: \textit{time$\_$space T} $\rightarrow \mathbb{U}$, dove U \'e la quantit\'a in questione. \\
\end{center}\end{tcolorbox}

Denoto una classe di segnali come:
$\mathcal{U}=\{u(\cdot):\mathcal{T} \rightarrow \mathbb{U}\}$

\section{Time space}
Per il time space $\mathcal{T}$ posso far riferimento a \textbf{continuous time, discrete event o discrete time}.

\subsection{Continuous Time Signals}
In continuous time descrivo quantit\'a fisiche. Un time space dev'essere totalmente ordinato, metrico (misurabile) e continuo (posso fare calcolo differenziale).
Tipicamente $\mathcal{T} \subset \mathbb{R}$. Lo uso se devo descrivere un circuitino ad esempio.

\subsection{Discrete Events Signals}
Nel caso di discrete events signals sono slegato dal tempo fisico, tutto quello che so \'e l'ordine degli eventi. 
Indipendentemente dal tempo fisico una sequenza in input deve dare lo stesso output. Il time space $\mathcal{T}$ dev'essere
ordinato e finito (tra due eventi devo aver un numero finito di altri eventi); pertanto uso $\mathbb{N}$. Lo uso per state transition diagrams.

\subsection{Discrete Time Signals}
Nel caso di discrete time signals ho una classe di discrete event signals con synchronous time instants. Tipicamente si sicronizzano 
gli istanti su un periodic time base. Il set dev'essere ordinato e abeliano (per poter computare somme e differenze di eventi).
Solitamente uso $\mathbb{Z}$.

Un sistema \'e formato da associazioni di sottosistemi eterogenei. 
DT (Discrete Time) systems sono ottenuti da CT (Continuous Time) systems con restrizioni temporali a quando certe quantit\'a si possono
misurare o certe variabili in input cambiano. 

Una quantit\'a fisica come il tempo viene campionata in molti periodi.
\begin{center}
\includegraphics[scale=0.35]{Chapters/Img/c01_01.png}\\
\end{center}

Solitamente uso calligraphic letter per denotare classi di segnali $\mathcal{U} = \{ \mathcal{u} (\cdot ) : \mathcal{T} \rightarrow \mathbb{U} \}$.
\begin{center}
    Time spaces:
    \begin{tabular}{lll}
        \textbf{Continuous time (CT)} signals & \textbf{Discrete Events (DE)} signals & \textbf{Discrete Time (DT)} signals\\
        totally ordered & totally ordered & totally ordered\\
        metric & fra due eventi posso averne finiti altri & sequenze di eventi associati a istanti \\ %temporali\\
        a continuum & & gli eventi devono essere \textbf{sincroni}\\
        & & gruppi abeliani \\
    \end{tabular}
\end{center}

CT si usa per rappresentare l'evoluzione di quantit\'a fisiche pertanto solitamente uso $\mathbb{R}$.

Per discrete events la storia \'e diversa, mi interessa solo l'ordine degli input, quanto tempo passa fra un input e l'altro non cambia l'output, solitamente uso 
$\mathbb{N}$. Ho una sequenza di eventi totalmente ordinati che posso mappare su $\mathbb{N}$.

Avendo DT eventi mappati su gruppi abeliani uso $\mathbb{Z}$.

Infine ho sistemi che sono collection di sottosistemi ogniuno con potenzialmente differenti time spaces.

\chapter{Sistemi}
\begin{tcolorbox}\begin{center}
    Considero $\mathcal{U}$ la classe dei segnali di input e $\mathcal{Y}$ la classe dei segnali di output che prendono valore da $Y$. Allora posso definire 
    il \textbf{sistema} come una relazione binaria fra $\mathcal{U}$ e $\mathcal{Y} :\ S \subset \mathcal{U} \otimes \mathcal{Y}$ (una relazione binaria sar\'a un set di 
    coppie (input, output)). 
\end{center}\end{tcolorbox}

Naturalmente posso avere lo stesso output per diversi input e viceversa (se, ad esempio, cambio lo stato iniziale).\\[5pt]

Denoto:
\begin{center}
    \begin{tabular}{ll}
        $\mathcal{T}(t_0) = \{ t \in \mathcal{T}: t \geq t_0 \}$ & il sottoinsieme del time space contenente gli istanti.\\
        $\mathcal{W}^{T(t_0)} = \{ w_0(\cdot) : \forall t \geq t_0,\ t \rightarrow w_0(t) \in W$ & il set di funzioni definite su $T(t_0)$ a valori in W.\\
        $w_0|_{T(t_1)}$ & la truncation della funzione che valuto da $t_1 > t_0$.\\
    \end{tabular}
\end{center}

\begin{tcolorbox}\begin{center}
    An \textbf{abstract dynamic system} is a 3-tuple $\{ T,\ \mathcal{U} \otimes \mathcal{Y},\ \sum \}$ con \\[5pt]
    $\sum = \{ \sum (t_0) \subset \mathcal{U}^{T(t_0)} \otimes \mathcal{Y}^{T(t_0)}: t_0 \in \mathcal{T}\ AND\ CRT\ is\ satisfied\}$ \\[5pt]
    dove CRT \'e la chiusura rispetto alla truncation (i.e. $\forall\ t_1 \geq t_0$).
\end{center}\end{tcolorbox}
$(u_0,\ y_0) \in \sum (t_0) \implies (u_0|_{T(t_1)},\ y_0|_{T(t_1)}) \in \sum (t_1)$, in pratica CRT property significa che se una coppia di funzioni appartiene al sistema 
da $t_0$ in poi, esse vengono troncate da $t_1$ in poi.



%\subsection{Esempi Linguaggi Liberi}
Essendo un linguaggio libero chiuso rispetto alla concatenazione, dati:\\
$L_1 = \{a^nb^nc^j \ / \ n, j \geq 0\} $ Libero\\
$L_2 = \{a^nb^nc^n \ / \ n,j \geq 0\}$ Libero perch\'e concatenazione di $\{a^nb^n\ / \ n \geq 0 \}$ e $\{ c^j \ / \ j \geq 0 \}$, entrambi liberi\\
$L_3 = \{a^nb^nc^n \ / \ n \geq 0\}$ Non \'e libero:

Suppongo $L_3$ libero, sia $p \in \mathbb{N}^+$, $z = a^pb^pc^p$
Allora $z \in  L_3,\ |z|=3p>p$\\
Spacco z in $A=a...a,\ B = b...b, C=c...c$\\
Siano $z = uvwxy\ \land\ |vwx| \leq p \ \land\ |vx| > 0 $:
\begin{itemize}
    \item vwx \'e composto da sole a in A\\
    \item vwx \'e composto da a in A e b in B\\
    \item vwx \'e composto da sole b in B\\
    \item vwx \'e composto da b in B e c in C\\
    \item vwx \'e composto da sole c in C\\
\end{itemize} 
Considero la parola $z' = uv^0wx^0y$ 
\begin{itemize}
    \item[1.] $z' = a^kb^pc^p,\ k<p,\ z' \not\in L_3 $\\
    \item[3.] $z' = a^pb^kc^p,\ k<p,\ z' \not\in L_3 $\\
    \item[5.] $z' = a^pb^pc^k,\ k<p,\ z' \not\in L_3 $\\
    \item[2.] $z' = a^kb^jc^p,\ k<p\ \lor\ j < p ,\ z' \not\in L_3 $\\
    \item[4.] $z' = a^pb^kc^j,\ k<p\ \lor\ j < p ,\ z' \not\in L_3 $\\
\end{itemize}
Quindi visto che la parola non appartiene mai ad $L_3$ il linguaggio non \'e libero.
$\Box$\\[5pt]

\begin{tcolorbox}\begin{center}
    Quindi la classe di linguaggi liberi \textbf{non \'e chiusa rispetto all'intersezione}
\end{center}\end{tcolorbox}

$L_4 = \{a^nb^mc^{n+m} \ / \ n,m>0\}$ Libero\\
$S \rightarrow aSc | aBc$\\
$B \rightarrow bBc|bc$\\

$L_5 = \{ a^nb^mc^nd^m | n,m > 0\}$ Non libero \\
$L_6= \{ wcw^R \ / \ w \in \{a,b\}^+\}$ Libero \\
$S \rightarrow aSa | bSb | aca|bcb$

%%%%%%%%%%%%%%%%%%%%%%%%%%%%%%%%%%%%%%%%%%%%%%%%%%%%%%%%%%%%%%%%%%%%%%%%%%%%%%%%%%%%%%%%%%%%%%%
\chapter{Automi a stati finiti}

Un NFA accetta/riconosce un certo linguaggio.

Sia N un NFA, allora il linguaggio riconosciuto/accettato da N \'e il set delle parole per le quali esiste almeno un cammino dallo stato iniziale di N ad uno stato finale di N.

notare che $\forall\ a \in A,\ a\epsilon = \epsilon a = a$.

%%%%%%%%%%%%%%%%%%%%%%%%%%%%%%%%%%%%%%%%%%%%%%%%%%%%%%%%%%%%%%%%%%%%%%%%%%%%%%%%%%%%%%%%%%%%%%%
\section{Thompson construction}
\begin{center}
    \begin{tabular}{ll}
        input & regular expression r\\
        output & NFA N $\ / \ L(N) = L(r)$\\ 
    \end{tabular}
\end{center}
Gli NFA usati nei passi della costruzione hanno:
\begin{itemize}
    \item un solo stato finale\\ 
    \item non hanno archi entranti sul nodo iniziale\\ 
    \item non hanno archi uscenti dal nodo finale\\
\end{itemize}

\textbf{Lemma}: Lo NFA ottenuto dalle costruzini di Thompson ha al massimo 2k stati e 4k archi, con k lunghezza della re. r.
\textbf{Osservazione}: Ogni passo della costruzione introduce al massimo 2 nodi e 4 archi.

\begin{center}
	\includegraphics[scale=0.4]{Chapters/Img/c02_01.png}\\
\end{center} 
Algoritmo a complessit\'a $O(|r|)$ 

%%%%%%%%%%%%%%%%%%%%%%%%%%%%%%%%%%%%%%%%%%%%%%%%%%%%%%%%%%%%%%%%%%%%%%%%%%%%%%%%%%%%%%%%%%%%%%%
\section{Simulare un NFA}
Il backtracking consiste nel seguire un percorso e se non va bene tornare in dietro e provarne un altro finch\'e alla fine 
li provo tutti mal che vada.

$N=(S,A,move_n,S_0,F)$, S insieme stati, A degli archi, $S_0$ stato iniziale, F set stati finali, $move_n$ funzione di transizione\\
$t \in S, T \subset S$\\
$\epsilon - closure(\{ t \})$ il set degli stati S raggiungibili tramite \underline{zero o pi\'u} 
$\epsilon -transizioni$ da t (in pratica il nodo stesso e tutti i nodi raggiungibili con una $\epsilon-transition$).

Nota che $\forall t \in S,\ t\in \epsilon-closure(t)$\\
$\epsilon-closure(T) = \cup _{t \in T} \epsilon-closure(t)$

Questo algoritmo \'e pi\'u performante del backtracking.

%%%%%%%%%%%%%%%%%%%%%%%%%%%%%%%%%%%%%%%%%%%%%%%%%%%%%%%%%%%%%%%%%%%%%%%%%%%%%%%%%%%%%%%%%%%%%%%
\subsection{Algoritmo per la computazione}
Strutture dati:
\begin{itemize}
    \item pila\\
    \item bool[] alreadyOn, dimensione $|S|$\\
    \item array[][] $move_n$\\ 
\end{itemize}
\begin{lstlisting}
    for(int i = 0; i < |S|; i++){
        alreadyOn[i] = false;
    }
    closure(t, stack){
        push t onto stack;
        alreadyOn[t] = true; //posso sempre arrivare a me stesso con una epsilon-transition 
        foreach(i in move_n(t, epsilon)){
            if(!alreadyOn[i]){
                closure(i, stack);
            }
        }
    }
\end{lstlisting}

\begin{center}
	\includegraphics[scale=0.5]{Chapters/Img/c02_02.png}\\
\end{center} 

\begin{lstlisting}
    alredyOn[F F F F];
    closure(A, pila vuota)
        [A] [T F F F]
            //B non e' ancora nella pila
            closure(B, [A])
                [A, B] [T T F T]
                closure(D, [A, B])
                    [A, B, D] [T T F T]
            closure(C, [A, B, D])
                [A, B, C, D] [T T T T]
\end{lstlisting}

%%%%%%%%%%%%%%%%%%%%%%%%%%%%%%%%%%%%%%%%%%%%%%%%%%%%%%%%%%%%%%%%%%%%%%%%%%%%%%%%%%%%%%%%%%%%%%%
\subsection{Algoritmo per la simulazione di un NFA}
\begin{center}
    \begin{tabular}{ll}
        input & NFA N, w$\$$\\
        output & yes se $w \in L(N)$, no altrimenti\\ 
    \end{tabular}
\end{center}
\begin{lstlisting}
    N = (S, A, move_n, S_0, F)
    states = epsilon-closure({S_0})
    symbol = nextchar()
    while(symbol != $){
        states = epsilon-closure(Unione_{t in states} di move_n(t, symbol));
        symbol = newxtchar();
    }
    if(states intersecato F != emptyset){
        return yes;
    }
    return no;
\end{lstlisting}

%TODO GRAFO PAGINA 17

Algoritmo a complessit\'a $O(|w|(n+m))$ 


%%%%%%%%%%%%%%%%%%%%%%%%%%%%%%%%%%%%%%%%%%%%%%%%%%%%%%%%%%%%%%%%%%%%%%%%%%%%%%%%%%%%%%%%%%%%%%%
\section{DFA}
Automa a stati finiti, deterministico; una sottoclasse degli NFA che rispettano:
\begin{center}
    DFA$\overset{\Delta}{=}(S,A,move_d,s_0,F)$  \\
    $move_d \overset{\Delta}{=} (S \otimes A) \rightarrow S$\\
\end{center}
\begin{itemize}
    \item non hanno $\epsilon-transizioni$\\
    \item $\forall a \in A, s \in S,\ move_n(s,a)$ \'e un unico stato se \textbf{funzione di transizione totale} 
    (al pi\'u uno stato se \textbf{funzione di transizione parziale})\\
\end{itemize}
\begin{tcolorbox}\begin{center}
    Sink è il nodo pozzo dove confluiscono tutte le transizioni non segnate;
    viene aggiunto per rendere la funzione di transizione una funzione di transizione totale
\end{center}\end{tcolorbox}

%%%%%%%%%%%%%%%%%%%%%%%%%%%%%%%%%%%%%%%%%%%%%%%%%%%%%%%%%%%%%%%%%%%%%%%%%%%%%%%%%%%%%%%%%%%%%%
\subsection{Linguaggio riconosciuto dal DFA}
Dato il DFA D, L(D) \'e il linguaggio riconosciuto da D. \\
$L(D) = \{ w=a_1,...,a_k \ / \ \exists \text{ cammino in D dallo stato iniziale al finale}\}$.
$\epsilon \in L(D) \iff s_0 \in F$.

%%%%%%%%%%%%%%%%%%%%%%%%%%%%%%%%%%%%%%%%%%%%%%%%%%%%%%%%%%%%%%%%%%%%%%%%%%%%%%%%%%%%%%%%%%
\subsection{Simulazione di un DFA con $move_d$ totale}
\begin{center}
    \begin{tabular}{ll}
        input & w$\$$, DFA $D=(S,A,move_d,F)$\\
        output & yes se $w \in L(D)$, no altrimenti\\
    \end{tabular}
\end{center}

\begin{lstlisting}
    state = s_0;
    while(symbol != $ && state != bottom){
        //move_d(s, a) = bottom <=> move_d non e' definita su (s,a)
        state = move_d(state, symbol);
        symbol = newxtchar();
    }
    if(state \in F)
        return yes;
    return true;
\end{lstlisting}

Simulazione NFA costa $O(|w|(n+m))$
Simulazione DFA costa $O(|w|)$

%%%%%%%%%%%%%%%%%%%%%%%%%%%%%%%%%%%%%%%%%%%%%%%%%%%%%%%%%%%%%%%%%%%%%%%%%%%%%%%%%%%%%%%%%%
\section{Subset Construction}
\begin{center}
    \begin{tabular}{ll}
        input & $NFA(S^n,A,move_n, S_0 ^n,F^n)$\\ 
        output & $DFA(S^d,A,move_d, S_0 ^d,F^d)$\\
    \end{tabular}
\end{center}

\begin{lstlisting}
	S_0^d = epsilon-closure({S_0^n});	
    //raggruppo stati della epsilon closure in un unico stato S_0^d del DFA
	states = {S_0^d};
	tag S_0^d come non marcato;

	while(exist T in states non marcato){
		marco T;
		foreach(a in A){ //guardo ogni arco
			T_1 = epsilon-closure(U_{t in T} di move_n(t,a));
            //tutti gli stati raggiungibili con una a-transition da uno stato in T 
            //poi la loro epsilon closure
			if(T_1 != emptySet){
				move_d(T, a) = T_1;
		 		if(T_1 !in states){
					aggiungi T_1 a states come non marcato;
				}
			}
		}
	}

	foreach(T in states){
		if( (T intersecato F^n) != 0){
			metti T_1 in F^d;
		}
	}
\end{lstlisting}
Lo stato iniziale del DFA sar\'a la $\epsilon - closure$ dallo stato iniziale del NFA (quindi un set di stati).
Considero lo stato iniziale del NFA e lo marco in grassetto poi espando $T_0$ con la $\epsilon - closure$ dello stato iniziale. 

Dallo stato $T_0$ guardo per ogni arco gli stati in cui arrivo e li marco in grassetto ($T_1$, $T_2$, ...); 
poi espando quelli in grassetto guardando le rispettive $\epsilon - closure$.

Alla fine guardo i set degli stati se due set coincidono mergio gli stati.

%%%%%%%%%%%%%%%%%%%%%%%%%%%%%%%%%%%%%%%%%%%%%%%%%%%%%%%%%%%%%%%%%%%%%%%%%%%%%%%%%%%%%%%%%%
\subsection{Esercizio}
\begin{center}
	\includegraphics[scale=0.5]{Chapters/Img/c02_05.png}\\
\end{center} 

\begin{tabular}{lll}
    \textbf{States}                                 &   \textbf{a}        &     \textbf{b} \\
    T0 = $\{$ \textbf{A} B C E $\}$                 &   T1                &     T2  = T0\\
    T1 = $\{$ A B C \textbf{D E} $\}$               &   T1                &     T0 \\
    T2 = $\{$ \textbf{A} B C \textbf{E} $\}$ = T0 (quindi T0 va in T0 tramite b) & come T0 & come T0 \\
\end{tabular}

%%%%%%%%%%%%%%%%%%%%%%%%%%%%%%%%%%%%%%%%%%%%%%%%%%%%%%%%%%%%%%%%%%%%%%%%%%%%%%%%%%%%%%%%%%
\subsection{Esercizio}
\begin{center}
	\includegraphics[scale=0.5]{Chapters/Img/c02_03.png}\\
\end{center} 

\begin{tabular}{lll}
    \textbf{States}                                 &   \textbf{a}        &     \textbf{b} \\
    $S_0^d = \{$ \textbf{0} 1 2 4 7 $\}$            &   T1                &     T2 \\
    T1 = $\{$ 1 2 \textbf{3} 4 6 7 \textbf{8} $\}$  &   T1                &     T3 \\
    T2 = $\{$ 1 2 4 \textbf{5} 6 7 $\}$             &   T1                &     T2 \\
    T3 = $\{$ 1 2 4 \textbf{5} 6 7 9 $\}$           &   T1                &     T4 \\
    T4 = $\{$ 1 2 4 \textbf{5} 6 7 \textbf{10} $\}$ &   T1                &     T2 \\
\end{tabular}

\begin{center}
	\includegraphics[scale=0.5]{Chapters/Img/c02_04.png}\\
\end{center} 

%%%%%%%%%%%%%%%%%%%%%%%%%%%%%%%%%%%%%%%%%%%%%%%%%%%%%%%%%%%%%%%%%%%%%%%%%%%%%%%%%%%%%%%%%%
\section{Partition Refinement}
\begin{tcolorbox}\begin{center}
    Guado gli archi, se tutta partizione punta ad un nodo dell'altra transizione con lo stesso non terminale allora va bene; 
    altrimenti spacco la partizione.
\end{center}\end{tcolorbox}

%%%%%%%%%%%%%%%%%%%%%%%%%%%%%%%%%%%%%%%%%%%%%%%%%%%%%%%%%%%%%%%%%%%%%%%%%%%%%%%%%%%%%%%%%%
\subsection{Algoritmo di Partition Refinement}
\begin{center}
    \begin{tabular}{ll}
        Input   &   DFA D = $\{ A,A,move_d, s_0, F\}$\\
        Output  &   partizione di S in blocchi equidistanti\\ 
    \end{tabular}
\end{center}

\begin{lstlisting}
    B_1 = F;
    B_2 = S \ F;
    P = {B_1, B_2};
    while(exists B_i, B_j in P, exists a in A, B_i e'' partizionabile rispetto a (B_j, a)){
        sostituire B_i in P con split(B_i, (B_j, a));
    }
\end{lstlisting}

%%%%%%%%%%%%%%%%%%%%%%%%%%%%%%%%%%%%%%%%%%%%%%%%%%%%%%%%%%%%%%%%%%%%%%%%%%%%%%%%%%%%%%%%%%
\subsection{Esempio}

\begin{center}
	\includegraphics[scale=0.5]{Chapters/Img/c02_04.png}\\
\end{center} 

\begin{tabular}{ll}
    $\{$ A B C D $\}$ $\{$ E $\}$                       & Considero le partizioni dei terminali e non terminali \\  
    $\{$ A B C D $\}$ $\{$ E $\}$                       & Con a-transizione non esco dal primo set\\  
    $\{$ A B C $\}$ $\{$ D $\}$ $\{$ E $\}$             & Con b-transizione vado da D in E (e A B C non vanno in E con b-transizioni)\\  
    $\{$ A B C $\}$ $\{$ D $\}$ $\{$ E $\}$             & Con a-transizione non esco\\    
    $\{$ A C $\}$ $\{$ B $\}$ $\{$ D $\}$ $\{$ E $\}$   & Con b-transizione vado da B in D e gli altri no quindi splitto \\
    $\{$ A C $\}$ $\{$ B $\}$ $\{$ D $\}$ $\{$ E $\}$   & vanno bene \\
\end{tabular}\\[5pt]

Rinomino  $\{$ A C $\}$ $\{$ B $\}$ $\{$ D $\}$ $\{$ E $\}$ in $t_1,\ t_2,\ t_3,\ t_4 $

\begin{center}
	\includegraphics[scale=0.5]{Chapters/Img/c02_06.png}\\
\end{center} 

%%%%%%%%%%%%%%%%%%%%%%%%%%%%%%%%%%%%%%%%%%%%%%%%%%%%%%%%%%%%%%%%%%%%%%%%%%%%%%%%%%%%%%%%%%
\section{Algoritmo di minimizzazione di DFA}
\begin{center}
    \begin{tabular}{ll}
        Input   &   DFA \textbf{D} = $\{ A,A,move_d, s_0, F\}$ con $move_d$ totale\\
        Output  &   minimo DFA (\textbf{min(D)}) che riconosce lo stesso linguaggio del primo\\ 
    \end{tabular}
\end{center}

\begin{lstlisting}
    P = PartitionRefinement(DFA D);
    // P = (B_1, ..., B_k);
    foreach(B_i in P){
        var t_i;            //do un nome alla partizione
        if(s_o in B_i){
            t_i e'' iniziale per min(D);    //setto lo stato iniziale di min(D)
        }
    }

    foreach(B_i in P, B_i in F){
        t_i e'' lo stato finale di min(D);  //setto lo stato finale di mind(D)
    }

    foreach( (B_i, a, B_j) tale che esiste s_i in B_i tali che move_d(s_i, a) = s_j){
        //per ogni tupla (stato, arco, stato) faccio la rispettiva transizione in min(D)
        setto una transizione temporanea in min(D) da t_i a t_j secondo il simbolo a;
    }

    foreach(dead state t_i){
        rimuovere t_i e tutte le transizioni da/verso t_i;
    }
    tutti i temporanei residui (sia stati che transizioni) sono gli stati e le transizioni di min(D);
\end{lstlisting}
Complessit\'a $O(nlgn)$.

%%%%%%%%%%%%%%%%%%%%%%%%%%%%%%%%%%%%%%%%%%%%%%%%%%%%%%%%%%%%%%%%%%%%%%%%%%%5
\subsection{Esempio}

\begin{center}
	\includegraphics[scale=0.5]{Chapters/Img/c02_06.png}\\
\end{center} 
Arrivato qua: rinominati $\{$ A C $\}$ $\{$ B $\}$ $\{$ D $\}$ $\{$ E $\}$ in $t_1,\ t_2,\ t_3,\ t_4 $,
applico la minimizzazione del DFA.

\begin{center}
	\includegraphics[scale=0.5]{Chapters/Img/c02_07.png}\\
\end{center} 

Ho le partizioni $\{A,\ B\}$, $\{C,\ sink\}$, applico partition refinement ma sono gi\'a partizionati correttamente.

\begin{center}
	\includegraphics[scale=0.5]{Chapters/Img/c02_08.png}\\
\end{center} 
\chapter{Linguaggi Regolari o Lineari}

%%%%%%%%%%%%%%%%%%%%%%%%%%%%%%%%%%%%%%%%%%%%%%%%%%%%%%%%%%%%%%%%%%%%%%%%%%%%%%%%%%%%%%%%%%%%%%%%%%%%%%%%%%%%%%%%%%%
\section{Da DFA a Grammatica Regolare}

Una grammatica \'e regolare se le produzioni sono della forma: $A \rightarrow \beta$, con $\beta$ terminale non-terminale, viceversa o 
terminale e basta.\\[5pt]

\begin{center}
    \begin{tabular}{lll}
        $A \rightarrow aB$     &    $B \rightarrow b$   &   grammatica lineare destra\\
        $B \rightarrow Ab$     &    $A \rightarrow a$   &   grammatica lineare sinistra\\
    \end{tabular}
\end{center}

\begin{tcolorbox}\begin{center}
    In pratica \'e la diretta trascrizione di un DFA in regex!
\end{center}\end{tcolorbox}

Dato un DFA D voglio trovare una grammatica regolare G tale che L(G) = L(D). 
Se ho una transizione $A \rightarrow B$ con una a-transizione diventer\'a $A \rightarrow aB$. Segno il nome del nodo che sto considerando
prima della freccia e, dopo la freccia, il non terminale ed il nodo destinazione. Se ho un nodo foglia C avr\'o  $C \rightarrow \varepsilon$.

Se invece ho una grammatica regolare e voglio trovare un DFA D $/\ L(G) = L(D)$, faccio il procedimento inverso a prima;
se ottengo un NFA basta fare Subset Construction.

%%%%%%%%%%%%%%%%%%%%%%%%%%%%%%%%%%%%%%%%%%%%%%%%%%%%%%%%%%%%%%%%%%%%%%%%%%%%%%%%%%%%%%%%%%%%%%%%%%%%%%%%%%%%%%%%%%%5
\subsection{Esempio}
$L=\{ w \ / \ w \in \{ a, b \} ^* \&\& |a|\ pari,\ |b|\ dispari \}$, L \'e regolare?
\begin{center}
	\includegraphics[scale=0.6]{Chapters/Img/c02_12.png}\\
\end{center} 
S\'i \'e regolare.

%%%%%%%%%%%%%%%%%%%%%%%%%%%%%%%%%%%%%%%%%%%%%%%%%%%%%%%%%%%%%%%%%%%%%%%%%%%%%%%%%%%%%%%%%%%%%%%%%%%%%%%%%%%%%%%%%%%5
\subsection{Considerazioni}
\begin{tcolorbox}\begin{center}
    Regular expression, NFA e DFA hanno la stessa potenza espressiva, sono solo notazioni diverse.\\
\end{center}\end{tcolorbox}

\begin{tcolorbox}\begin{center}
    Dal DFA posso sempre costruirmi una \textbf{grammatica regolare} equivalente.\\
\end{center}\end{tcolorbox}

\begin{tcolorbox}\begin{center}
    Non devo fare l'errore di assumere che qualsiasi grammatica sia esprimibile attraverso un NFA.\\
\end{center}\end{tcolorbox}
%%%%%%%%%%%%%%%%%%%%%%%%%%%%%%%%%%%%%%%%%%%%%%%%%%%%%%%%%%%%%%%%%%%%%%%%%%%%%%%%%%%%%%%%%%%%%%%%%%%%%%%%%%%%%%%%%%%5
\subsection{Esempio}
$L=\{ w \ / \ w \in \{ a, b \} ^* \&\& |a|=|b|\}$, L \'e regolare?
\begin{center}
	\includegraphics[scale=0.6]{Chapters/Img/c02_13.png}\\
\end{center} 
Non potr\'a mai essere regolare, per il pumping lemma per i linguaggi regolari.

%%%%%%%%%%%%%%%%%%%%%%%%%%%%%%%%%%%%%%%%%%%%%%%%%%%%%%%%%%%%%%%%%%%%%%%%%%%%%%%%%%%%%%%%%%%%%%%
\section{Pumping Lemma per Linguaggi Regolari}

Sia L un linguaggio regolare
$\implies \exists\ p \in \mathbb{N}^+ \ / \ \forall\ z \in L \ / \ |z| > p$,
$\exists\ u,v,w \ / \ $:

\begin{itemize}
    \item[i)] $z = uvw\ \land$\\
    \item[i)] $|uw| \leq p\ \land$\\
    \item[i)] $|v| > 0,\ \forall i \in \mathbb{N},\ u v^i w \in L$\\
\end{itemize}

%%%%%%%%%%%%%%%%%%%%%%%%%%%%%%%%%%%%%%%%%%%%%%%%%%%%%%%%%%%%%%%%%%%%%%%%%%%%%%%%%%%%%%%%%%%%%%%
\subsection{Dimostrazione}
L \'e regolare quindi pu\'o essere riconosciuto da un automa a stati finiti.\\
Sia D il min DFA $\ / \ L(D) = L,\ p = |S|$, allora i cammini pi\'u lunghi che non passano pi\'u di una volta nel medesimo 
stato hanno al pi\'u lunghezza (p-1).\\
Allora se $z \in L$ con $|z|>p$, z \'e riconosciuta tramite un cammino che attraversa almeno due volte uno stato. \\

%%%%%%%%%%%%%%%%%%%%%%%%%%%%%%%%%%%%%%%%%%%%%%%%%%%%%%%%%%%%%%%%%%%%%%%%%%%%%%%%%%%%%%%%%%%%%%%%%%%%%%%%%%%%%%%%%%%5
\subsection{Negazione testi Pumping Lemma per linguaggi regolari}
$\forall p \in \mathbb{N}^+ \ / \ \exists\ z \in L \ / \ |z| > p.\ \forall\ uvw $
$z = uvw \land |uw| \leq p \land |v| > 0) \implies \exists\ i \in \mathbb{N} \ / \ u v^i w \not\in L) $

Lemma: $L=\{a^nb^n \ / \ n \geq 0\}$ non \'e regolare\\
Dim: Assumo per assurdo che L sia regolare, dato p un qualunque numero positivo e $z= a^p b^p $ allora
$\forall\ uvw \ / \ z = uvw \land |uw| \leq p \land |v| > 0$ (la stringa v contiene solo (e almeno una) \lq a\rq ).

allora $uv^2w$ ha la forma $a^{p+k}b^p,\ k > 0$
allora $uv^2w \not\in L$ il che contraddice il Pumping Lemma per linguaggi regolari.

[v pu\'o assumere $a^i$, $a^i b^j$, $b^i$, in ogni caso per qualunque potenza di v non appartiene ad L 
(con $(a^i b^j)^2$ ho )]

%%%%%%%%%%%%%%%%%%%%%%%%%%%%%%%%%%%%%%%%%%%%%%%%%%%%%%%%%%%%%%%%%%%%%%%%%%%%%%%%%%%%%%%%%%%%%%%%%%%%%%%%%%%%%%%%%%%5
\subsection{Esercizio}
$L_1 = \{ w \ / \ w \in \{ a, b \}^* \text{e contiene almeno una occorrenza di \lq\lq aa \rq\rq}\}$\\

$L_1$:
$A \rightarrow aA | bA | aB$\\
$B \rightarrow aC$\\
$C \rightarrow aC | bC | \varepsilon $\\

$L_2 = \{ ww \ / \ \ \in \{ a,b \}^* \}$\\
non \'e libero per il pumping lemma (gi\'a dimostrato), quindi non \'e regolare.

\begin{tcolorbox}\begin{center}
    $\neg $ L Libero $\implies \neg $ L Regolare\\
    $\neg $ L Libero $\centernot\iff \neg $ L Regolare
\end{center}\end{tcolorbox}

$L_3 = \{ww^r \ / \ w \in \{ a,b \}^* \}$\\
Non \'e regolare ma libero.
$z = a^p b^p b^p a^p \in L_3$
visto che $uv < p$, uv \'e composta solo da a 
$u v^i w = a^pb^{2p}a^p \not\in L_3$ quindi non pu\'o essere regolare.

[$w^r$ \'e w rovesciato]
%%%%%%%%%%%%%%%%%%%%%%%%%%%%%%%%%%%%%%%%%%%%%%%%%%%%%%%%%%%%%%%%%%%%%%%%%%%%%%%%%%%%%%%%%%%%%%%%%%%%%%%%%%%%%%%%%%%5
\subsection{Esercizi di esame}
Sia $N_1$ lo NFA con stato iniziale A e finale E con la seguente funzione di transizione:

\begin{center}
    \begin{tabular}{|c|c|c|c|}
        \hline
            &   $\varepsilon$      &   a           &   b           \\
        \hline
            A    &   $\{B, E\}$ &   $\o$        &  $\o$         \\
        \hline
            B    &   $\{ C \}$  &   $\o$        &  $\{ E \}$    \\
        \hline
            C    &   $\o$       &   $\{ D \}$   &  $\o$         \\
        \hline
            D    &   $\{ E \}$  &   $\o$        &  $\{ B \}$    \\
        \hline
            E    &   $\o$       &    $\{ E \}$  &  $\{ A \}$    \\
        \hline
    \end{tabular}
\end{center} 

\begin{itemize}
    \item[1)] $aa \in L(N_1)$?\\
    \item[2)] D \'e il DFA ottenuto da $N_1$, per subset construction, Q stato iniziale di D, $Q_{ab\_}$ lo stato di D che si 
    raggiunge da Q tramite il cammino ab. Dire a quale sottoinsieme degli stati di $N_1$ corrisponde $Q_{ab\_}$.\\    
\end{itemize}

\begin{center}
	\includegraphics[scale=0.6]{Chapters/Img/c02_19.png}\\
\end{center} 

1) S\'i facendo $A \rightarrow B \rightarrow C \rightarrow D \rightarrow E \rightarrow E $
2) Facendo la subset construction:

\begin{center}
    \begin{tabular}{|l|l|l|}
        \hline
                                &   a   &  b    \\
        \hline
            $Q0 = \{A,B,C,D\}$  &   Q1  &  Q0   \\
        \hline
            $Q1 = \{D,E\}$      &   Q2  &  Q0   \\
        \hline
           $Q2 = \{E\}$         &   Q2  &  Q0   \\
        \hline
    \end{tabular}
\end{center} 

\begin{center}
	\includegraphics[scale=0.6]{Chapters/Img/c03_01.png}\\
\end{center} 
%\chapter{14/11/17}
LALR(1)
LR(1)

\section{LRm(1)}
\subsection{Esempio}
\begin{tabular}{l}
	$S \rightarrow L=R|R$\\
	$L \rightarrow *R|id$\\
	$R \rightarrow L $\\
\end{tabular}
Lo stato iniziale \'e associato ad una variabile; ottenuto come chiusura (considero gli item LR(1)) 
$S' \rightarrow .S_1\{x_0\}$. \\

\subsubsection{Procedo con la chiusura LR(1)} 
\begin{tabular}{l}
	Equazioni\\
	$x_0 \doteq \{ \$ \}$\\
	$x_1 \doteq \{ x_0 \}$\\
	$x_1 \doteq x_0$\\
	$x_2 \doteq x_0$\\
	$x_3 \doteq x_0$\\
	$x_4 \doteq x_0$\\
	$x_5 \doteq \{ =, x_0 \} \cup \{ x_5 \} \cup \{ x_7 \}$\\
	$x_6 \doteq \{ =, x_0 \} \cup \{ x_5 \} \cup \{ x_7 \}$\\
	$x_7 \doteq \{ x_2 \}$\\
	$x_8 \doteq \{ x_5 \}$\\
	$x_9 \doteq \{ x_5 \} \cup \{ x_7 \}$\\
	$x_10 \doteq \{ x_7 \}$\\
\end{tabular}

\begin{tabular}{l}
	0) Stato 0-esimo
	$S' \rightarrow .S, \{ x_0 \}$\\
	...........................\\
	$S \rightarrow .L=R,\ \{x_0\}$\\
	$S \rightarrow .R,\ \{x_0\}$\\
	$L \rightarrow .*R,\ \{=, x_0\}$\\
	$L \rightarrow .id,\ \{=, x_0\}$\\
	$R \rightarrow .L,\ \{x_0\}$\\
\end{tabular}

\begin{tabular}{l}
	1) Transizione su S ($\rightarrow S$)\\
	$S' \rightarrow S.,\{ x_1 \}$\\
\end{tabular}

\begin{tabular}{l}
	2) Transizione su L\\
	$S \rightarrow L.=R, \{ x_2 \}$\\
	$R \rightarrow L., \{ x_3 \}$\\
\end{tabular}

\begin{tabular}{l}
	3) Transizione su R\\
	$S \rightarrow R., \{ x_4 \}$\\
\end{tabular}

\begin{tabular}{l}
	4) Transizione su *\\
	$L \rightarrow *.R, \{ x_5 \}$\\
	..............................
	$R \rightarrow .L, \{ x_5 \}$\\
	$R \rightarrow .*R, \{ x_5 \}$\\
	$R \rightarrow .id, \{ x_5 \}$\\
\end{tabular}

\begin{tabular}{l}
	5) Transizione su id\\
	$L \rightarrow id., \{ x_6 \}$\\
\end{tabular}

\begin{tabular}{l}
	6) Transizione da 2 su =\\
	$S \rightarrow L=.R, \{ x_7 \}$\\
	...............................\\
	$R \rightarrow .L, \{ x_7 \}$\\
	$L \rightarrow .*R, \{ x_7 \}$\\
	$L \rightarrow .id, \{ x_7 \}$\\
\end{tabular}

\begin{tabular}{l}
	7) Transizione da 4 su R\\
	$L \rightarrow *R., \{ x_8 \}$\\
\end{tabular}

\begin{tabular}{l}
	8) Transizione da 4 su L\\
	$R \rightarrow L., \{ x_9 \}$\\
	Ho un loop sullo stato 4 su * (transizione da 4 in 4)\\
\end{tabular}

\begin{tabular}{l}
	9) Transizione da 6 su R\\
	$S \rightarrow L=R., \{ x_10 \}$\\
	Inserire una L transizione da 6 a 8\\
	Inserire una * transizione da 6 a 4\\
	Inserire una id transizione da 6 a 5\\
\end{tabular}

\begin{tabular}{l}
	Equazioni\\
	$x_0 \doteq \{ \$ \}$\\
	$x_1 \doteq \{ x_0 \}$\\
	$x_1 \doteq x_0$\\
	$x_2 \doteq x_0$\\
	$x_3 \doteq x_0$\\
	$x_4 \doteq x_0$\\
	$x_5 \doteq \{ =, x_0 \} \cup \{ x_5 \} \cup \{ x_7 \}$\\
	$x_6 \doteq \{ =, x_0 \} \cup \{ x_5 \} \cup \{ x_7 \}$\\
	$x_7 \doteq \{ x_2 \}$\\
	$x_8 \doteq \{ x_5 \}$\\
	$x_9 \doteq \{ x_5 \} \cup \{ x_7 \}$\\
	$x_10 \doteq \{ x_7 \}$\\
\end{tabular}

\begin{tcolorbox}\begin{center}
	Controllo le equazioni dalla prima all'ultima e guardo se l'inisieme di destra ho una sola variabile o due di cui una il nodo di destra (self loop).
\end{center}\end{tcolorbox}

	$x_j = \{ x_i \} \cup \{ x_j \} = \{ x_i \}$\\
\begin{tabular}{ll}
	equazione & classe di equivalenza\\
	$x_0 = \{ \$ \}$ & $x_0$\\
	$x_1$ & $x_0$\\
	$x_2$ & $x_0$\\
	$x_3$ & $x_0$\\
	$x_4$ & $x_0$\\

	$x_5 = \{ =, x_0, x_5, x_7 \}$ & $ x_5 $\\
	$x_6 = \{ =, x_0, x_5, x_7 \}$ & $x_0$\\
	$x_7 = \{ \$ \}$ & $ x_6 $\\
	$x_8$ & $x_5$\\
	$x_9$ & $x_0$\\
	$x_10$ & $x_5$\\
\end{tabular}

\begin{tabular}{l}
	$x_0 = \$ $\\
	$x_5 = \{ =, x_0, x_5, x_7 \}$\\
	$x_6 = \{ =, x_0, x_5, x_7 \}$\\
	$x_9 = \{ x_5, x_7 \}$\\
\end{tabular}

\begin{tabular}{l}
	Tolgo i self loop e ridondanze
	$x_0 = \$ $\\
	$x_5 = \{ =, x_0 \}$\\
	$x_6 = \{ =, x_0, x_5 \}$ ($x_7 = x_0$, lo tolgo)\\
	$x_9 = \{ x_5, x_0 \}$\\
\end{tabular}

\begin{center}
	\includegraphics[scale=0.5]{Chapters/Img/l01_01.png}\\
\end{center} 

\begin{tabular}{l}
	$x_0 = \{ \$ \}$\\
	$x_5 = \{ =, \$ \}$\\
	$x_6 = \{ =, \$ \}$\\
	$x_9 = \{ =, \$ \}$\\
\end{tabular}

\section{Algoritmo}
Algoritmo utilizzato da yacc (su dragonbook)
Uso LR0, per ogni item dentro uno stato faccio una chiusura virtuale LR1
Faccio un nuovo grafo e ...

Identifica look ahead generati e propagati, poi fa un grafo e associa ad ogni nodo il look ahead generato

\begin{center}
	\includegraphics[scale=0.5]{Chapters/Img/l01_02.png}\\
\end{center} 

\begin{tcolorbox}\begin{center}
	Kernel(P) L'insiseme dei kernel item dentro P, proj(P) L'inisieme degli item LR(0) contenuti in P
\end{center}\end{tcolorbox}

\section{Costruzione automa simbolico}
\begin{lstlisting}
	x_0 = new Var()
	vars = {x_0}
	P_0 = closure_1( { [s' -> .s, {x_0} ] } ) //stato iniziale automa
	inizializzare Queue con x_0 = {$}
	States = {P_0}
	porre P_0 come unmarked
	while(c'e' uno stato non marcato P in States){
		marcare P
		foreach(y in V){
			Tmp = emptySet
			foreach([A -> alpha.y beta, delta] <- P){
				aggiungere [A -> alpha.y beta, delta] a Tmp
				if(Tmp != emptySet){
					if( lo stato targhet non e' ancora stato collezionato){
						aggiungere a States una versione simbolica del targhet e aggiungere a Queue una equazione per kernel item in Tmp
					} else {
						raffinare le equazioni delle variabili associate ai kernel item del target
					}
				}
			}
		}
	}
\end{lstlisting}


%\section{Grammatiche Attribute}
SDD (Syntax Directed Definition)
SDT (Syntax Directed Translation)

Aggiungo alla grammatica degli attributi e delle regole per quest'utlimi.
(Desc Calculator) posso computare valori associati ad espressioni aritmetiche.

%%%%%%%%%%%%%%%%%%%%%%%%%%%%%%%%%%%%%%%%%%%%%%%%%%%%%%%%%%%%%%%%%%%%%%%%%%%%
\subsection{Esempio}
\begin{tabular}{l}
	$E \rightarrow E + T | T$\\
	$T \rightarrow T * F | F$\\
	$F \rightarrow (E) | id$\\
\end{tabular}

Associo attributi ai terminali e non terminali, che recupero dall'analisi lessicale.

\begin{center}
	$3 + 4 * 5$ \\
	\Tree[. E [.E [.T [.F [.id (3) ] ] ] ] + [.T [.T [.F [.id (4) ] ] ] * [.F [.id (5) ] ] ] ]\\
\end{center}


Faccio visita postorder dell'albero (Bottom Up), associo i valori degli attributi a id, poi a F, poi risalgo a T. Quando sonon in T * F, T=4, F=5;
Quindi risolvo 4*5 che assegno come attributo al nodo padre T.

\begin{tcolorbox}\begin{center}
	\textbf{E.val} per indicare l'attributo della E.
\end{center}\end{tcolorbox}

\begin{tabular}{ll}
	$ \$\$ $ 	& Il driver\\
	$ \$ 1 $	& Primo elemento della produzione\\
\end{tabular}	 

$E_1 \rightarrow E_2 + T$ $\{ E_1.val = E_2.val + T.val \} $ \textbf{azione semantica} \\
$E \rightarrow T$ $\{ E.val = T.val \} $ \\
$T_1 \rightarrow T_2 * F $ $\{ T_1.val = T_2.val + F.val \} $ \\
$T \rightarrow F $ $ T.val = F.val $ \\
$F \rightarrow (E) $ $\{ F.val = E.val \} $ \\
$F \rightarrow id $ $\{ F.val = lexval(id) \} $ \\

Abstract Syntax Tree (albero derivazione \lq\lq ristretto\rq\rq )
\Tree[.+ $id_3$ [.* $id_4$ $id_5$ ] ]

\subsection{Attributi sintetizzati}
$A \rightarrow \alpha $ A.a definito come una funzione degli attributi dei terminali e non terminali in $\alpha $.
Gli attributi dei terminali derivano dall'analisi lessicale.

\subsection{Esempio, dichiarazione variabili}
\begin{lstlisting}
	int pippo, pluto, paperino;
\end{lstlisting}

$D \rightarrow TL$ $\{ L.i = T.t \} $\\
$T \rightarrow int$ $\{ T.t = integer\} $\\
$T \rightarrow float$ $\{ T.t = float \} $\\
$L_1 \rightarrow L_2,\ id$ $\{ L_2.i = L_1.i,\ addType(lex(id), L_1.i) \} $\\
$L \rightarrow id$ $\{ addType(lex(id)),\ L_i) \} $\\

\Tree[.D [.T int ] [.L [.L [.L [.id (pluto) ] ] , [.id (pippo) ] ] , [.id (paperino) ] ] ]

\begin{center}
	\includegraphics[scale=0.6]{Chapters/Img/l02_01.png}\\
\end{center} 

Gli attributi ereditati sono funzione degli attributi di siblings e del padre (driver della produzione).

\subsection{Example}
$S \rightarrow Number$\\
$Number \rightarrow o Digits$ serie di cifre in ottali (o)\\
$Number \rightarrow Digits d$\\
$Digits \rightarrow d$\\

\Tree[.N o [.Digits [.(...) d ] d ] ]

Gira l'albero!
$ S \rightarrow Digits $\\
$ Digits_1 \rightarrow Digits_2 d $ $\{ Digits_1.val = Digits_2 * Dg.tg_2.base + "d" \}$\\
$ Digits \rightarrow d $ $\{ Digit.base = 10; Digit.val = "d" \}$\\
$ Digits \rightarrow od $ $\{ Digit.base = 8; Digit.val = "d" \}$\\

Supponiamo per assurdo che sia LALR.
\end{document}