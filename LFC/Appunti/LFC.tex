\documentclass[a4paper,10pt,oneside,openany]{book}
\usepackage[utf8]{inputenc}
\usepackage{amsmath}
\usepackage{cancel}
\usepackage{amsfonts}
\usepackage{amssymb}
\usepackage{xcolor}
\usepackage{mathrsfs}
\usepackage{amsfonts}
\usepackage{amsmath}
\usepackage{mathtools}
\usepackage{centernot}
\usepackage{graphicx}
\usepackage{setspace}
\usepackage{geometry}
\usepackage{tcolorbox}
\usepackage{hyperref}
%%%%%%%%%%%%%%%%%%%%%%%%%%%%%%%%%%%%%%%%%%%%%%%%%%%%%%%%%%%%%%%%%%%%%%%%%%%%%%%%%%%%%%%%%%%%%%%%%%
\usepackage{multicol}

%%%%%%%%%%%%%%%%%%%%%%%%%%%%%%%%%%%%%%%%%%%%%%%%%%%%%%%%%%%%%%%%%%%%%%%%%%%%%%%%%%%%%%%%%%%%%%%%%%
\usepackage{listings}
\usepackage{color}
\usepackage{qtree}

\definecolor{dkgreen}{rgb}{0,0.6,0}
\definecolor{gray}{rgb}{0.5,0.5,0.5}
\definecolor{mauve}{rgb}{0.58,0,0.82}

\lstset{frame=tb,
  language=C++,
  aboveskip=3mm,
  belowskip=3mm,
  showstringspaces=false,
  columns=flexible,
  basicstyle={\small\ttfamily},
 %numbers=left,
  numbers=none,
  numberstyle=\tiny\color{gray},
  keywordstyle=\color{blue},
  commentstyle=\color{dkgreen},
  stringstyle=\color{mauve},
  breaklines=true,
  breakatwhitespace=true,
  tabsize=3,
  %xleftmargin=0.5cm
}
%%%%%%%%%%%%%%%%%%%%%%%%%%%%%%%%%%%%%%%%%%%%%%%%%%%%%%%%%%
\usepackage{graphicx}
\graphicspath{ {images/} }

\definecolor{grigio}{RGB}{179,179,179}
%\theoremstyle{plain}
\newtheorem{thm}{Teorema}[section]
\newtheorem{deff}{Definizione}[section]
\newtheorem{dimm}{Dimostrazione}[section]

\geometry{a4paper,top=1.5cm,bottom=1.5cm,left=2.5cm,right=2.5cm,%
heightrounded,bindingoffset=5mm}

\newcommand{\tb}[1]{\textbf{#1}}
\newcommand{\e}[1]{\textit{#1}}
\newcommand{\DEF}[1]{\textcolor{blue}{DEF:}}
\newcommand{\om}[0]{$\Omega$}
\newcommand{\al}[0]{$\mathcal{A}$}
\newcommand{\zero}[0]{$\emptyset$}
\newcommand{\ap}[0]{$\in$}
\newcommand{\im}[0]{$\implies$}

\setcounter{tocdepth}{1} %il grado di profondita indice, 1 solo chapter, 2 section ...

\author{Edoardo Lenzi}
\title{LFC (Linguaggi Formali e Compilatori) - Note del Corso}
\begin{document}
\maketitle
\tableofcontents
\chapter{Introduzione}
Un compilatore \'e un programma che legge un \textbf{linguaggio source} e lo traduce in un \texttt{equilvalente} \textbf{linguaggio di programmazione
target}. 
Solitamente il compilatore compila in \texttt{assembly} e poi un \textbf{assembler} produce codice macchina.
Se il target language \'e un programma eseguibile pu\'o processare input e produrre output.

Un \textbf{interprete} \'e un altro tipo di language processor, invece di tradurre il linguaggio lo esegue direttamente
quindi piglia sia il source program che gli input e processa l'output

Infine il \textbf{preprocessore} risolve le macro nel sorgente codificandole in linguaggio nativo (espandendole) prima di compilare.

\begin{quote}
Solitamente il compilato va pi\'u veloce mentre l`interprete ti da diagnosi piu accurate dato che esegue il codice.
Nel caso di Java compilo il sorgente in linguaggio intermedio \texttt{bytecode} che poi interpreto sulla JVM.
\end{quote}

Il \textbf{linker} \lq\lq linka\rq\rq\ assieme moduli e librerie dove ho riferimenti ad altri file (risolve gli indirizzi).
Il \textbf{loader} invece fa il merge in memoria per l'esecuzione.

%img4

\section{Front-End of the Compiler}
La \textbf{parte analitica} del processo di compilazione spacca la sorgente in parti costituenti e impone su di esse una struttura 
grammaticale (stile dtd); sfrutta questa struttura per creare una rappresentazione intermedia.
Se non passa la validazione grammaticale mi tira errori. Il sorgente viene storicizzato in
una struttura dati chiamata \textbf{symbol table}. 

\section{Back-End of the Compiler}
La \textbf{parte di sintesi} invece traduce il sorgente guardando la rappresentazione intermedia e la symbol table;
le parti di analisi e sintesi sono chiamate anche \textbf{front-end of the copiler} mentre le restanti \textbf{back-end}.

\section{Lexical analysis}
Fa uno scan e raggruppa le parole in \textbf{lexems}, per ogni lexem genera un \textbf{token} della forma

\begin{center}
    \texttt{(token name, attribute value)} 
\end{center}

Il \texttt{token name} \'e un simbolo astratto usato nella syntax analysis
mentre il \texttt{value} \'e un puntatore alla symbol table entry.\\[5pt]

\textbf{ie)}
\begin{quote}
    \texttt{position = initial + rate * 60} diventa \texttt{(id, 1) (=) (id, 2) (+) (id, 3) (*) (60)}\\
    gli operatori matematici sono simboli astratti che non hanno attribute value (?non sono referenziati nella symbol table?).
\end{quote}

%img7

\section{Session syntax analyzer} 
\'E un parsing, con i token crea una \textbf{rappresentazione ad albero (syntax tree)} nel quale 
il nodo \'e un operatore e i figli gli operandi. 

gli operatori devono avere priorit\'a per costruire l'albero, la struttura grammaticale serve anche a 
definire le priorit\'a degli operatori.

\section{Semantic analyzer} 
Piglia il \texttt{syntax tree} e guarda se \'e semanticamente consistente con la definizione del linguaggio.
(ie \texttt{type checking}). Il linguaggio puo ammettere cast impliciti chiamati \textbf{coercizioni} o tirare cogne.

\lq\lq \textit{intofloat}\rq\rq \'e una coercizione dell'intero 60 in float dato che gli altri operandi sono float. 

\section{Intermediate code generation}
Nel processo di compilazione posso avere varie rappresentazioni intermedie come alberi etc..
Dopo semantic analysis solitmente creo un codice basso livello, machine-like, \textit{\lq\lq easy to 
produce and esay to translate int target machine code\rq\rq}. Nella figura ho un tree address code
ricavato dal syntax tree. 

In un tree address code a destra ho al massimo un operatore (assembly like), e le operazioni sono in ordine.

Devo avere variabiline intermedie
\section{Code generation}
Segue la fase opzionale di \textbf{code optimization}, prende la rappresentazione intermedia e la mappa in un target language.
Le istruzioni intermedie vengono tradote in istruzioni macchina (presumibilmente). Devo capire come mappare variabili su registri

Nella symbol table devo storicizzare tutti gli attributi di un variable name.
 
Solitamente posso agglomerare le fasi di analisi in front end pass e le altre in back end pass.







\chapter{Progetto}
%https://sites.google.com/a/unitn.it/software-engineering-ii---designing-applications-that-matter/home
%https://github.com/marcorobol/is2-2017
%https://www.facebook.com/trentose2
%https://sites.google.com/site/sphoebss/project-definition
%https://sites.google.com/site/disibpm/

DOM struttura della pagina HTML 
JS mono-thread ma si comporta come multithread
devo installare nodeJs (verifico la presenza con \texttt{node -v}

Devo buttare tutto su heroku


\end{document}