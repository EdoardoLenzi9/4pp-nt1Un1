\section{Intro}
L'ingegneria del software \'e l'applicazione di un approccio disciplinato e 
quantificabile per migliorare lo sviluppo sw.

\'E la scienza del creare e mantenere sw utile, affidabile e sostenibile economicamente

Multiversion e multipersonal (principio di Parnas)

Empathize (capire le esigenze dell'utente finale)
Define (creare un punto di vista basato sui bisogni utente)
Ideate (brainstorm per le possibili soluzioni)
Prototype (creare una rappresentazione delle idee)
Test (share prototype)

DevOps (Developmetn Operations)
tutte le fasi, non un semplice lavorare al codice.

waterfall
comunication (inizio progetto, raccolta requisiti)
planning (stime, scheduling, tracking)
modeling (analysis, design)
construction (code, test)
deployment (delivery, support)

Agile 
sviluppare sw in incrementi di brevi intervalli temporali fissati
coinvolgere nello sviluppo il cliente da subito

API 
ogni team espone dati e funzionalit\'a tramite queste interfacce di servizi
i team si interfacciano sulle api, non sono legate ad una specifica tecnologia

%%%%%%%%%%%%%%%%%%%%%%%%%%%%%%%%%%%%%%%%%%%%%%%%%%%%%%%%%%%%%%%%%%%%%%%%%%%%
\section{Agile dev with scrum and kanban}

Agile prevede di soddisfare il cliente con consegne veloci e continue di 
software di qualit\'a'
Il cambio di requisiti anche a processo avviato non \'e un problema.
Rilasciare codice funzionante in poche settimane/mesi
Creare progetti attorno a individui motivati.

Codice che funziona \'e la prima misura di progresso

Ho gli sprint backlogs -> current sprint backloag-> backloag tasks expanded by the team 
vanno fatti un uno sprint di un mese e ogni giorno faccio un daily scrum meeting

I ruoli scrum sono owner, team (massimo sette persone), scrum master 
(protegge il team, managing processi, call meetings)

\subsection{User Stories}
Hanno un titolo, una breve desc (As a user I want ...) e una descrizione 
funzionale (how to demo) + importance e estimate

La storia viene spaccata in tasks 

\subsection{Sprint}
sono le iterazioni in scrum, da due settimane a due mesi. Alla fine 
devo avere sw funzionante e rilasciabile.

Le stories vengono prese dal backlog in ordine di priorit\'a.

Si fa uno sprint planning meeting in cui decido i goal dello sprint, il team,
le deadline e i daily meeting.

\subsection{Misurare la velocit\'a}
person days * focus factor = estimates velocity
focus factor = actual velocity / person days
(attorno al 70%)

\subsection{Testers}
I testers accedono al sistema come gli utenti finali, non fanno parte del 
core team, interagiscono con il sistema in modi che gli sviluppatori forse 
non avevano pensato pertanto trovano bug.

I membri del team hanno responsabilit\'a di amministratori del backlog e dello 
sprint. 

Nello daily standup meeting ogni membro risponde alle domande:
    cosa hai fatto? cosa farai? che problemi ci sono?

Mai estendere lo sprint, non va bene passare la deadline, non cambiare i goal
dello sprint. Nel caso siamo in ritarno evita di fare straordinari, mantieni un 
ritmo sostenibile. 

Alla fine faccio un meeting di retrospective dove tutti parlano dei pro e dei 
contro

kanban
limit WIP
measure lead time (misura tempo di consegna)

\chapter{testing}
\section{Equivalence Partitioning}
tecnica di testing sw che divide input e output di un modulo software in 
partizioni di dati sulle quali posso fare test.

socpre alcune classi di errori
socpre classi di input che generano errori
devo coprire casi speciali come null, out of range, blank, ...
potrei partizionare per intervalli

tendo a relazionarlo con black box testing (\'e praticamente la stessa cosa)
gray box technique as well:
ci sono ulteriori partizioni semantiche in base al flusso del codice

si basa un po' sul principio del boundary value analysis (considero i valori 
agli estremi/media di equivalence partitions)

nel caso del boundary value specifico dei valori di contorno per test cases.

Il testing consiste nell'esecuzione di un programma per scoprire bugs.

Un test case \'e un set di valori input e output attesi e precondizioni per 
eseguire un test con lo scopo di trovare errori.
Il test fallisce se expected result != actual result 

\section{Types of testing}
Black box testing \'e un metodo per esaminare le funzionalit\'a di un app.
senza andare nella sua struttura interna.

White box testing invece serve a testare la struttura interna dell'applicazione.
(opposto al testing delle funzionalit\'a).

Unit testing, serve a testare una funzionalit\'a in isolamento (non so 
come la funzionalit\'a verr\'a usata).

Integration testing, prendo molte funzioni (gi\'a testate) e le 
testo in combinazione

test di sistema, testo il sistema nella sua completezza

Acceptance test (servono per determinare se un requisito di contratto \'e 
stato portato a termine)

Regression test (runno i test vecchi su una nuova implementazione di un componente)

A/B testing per testare l'efficacia di un design

Per creare casi di test mi affido ad equivalence partitioning;

\section{Quanti test servono}
Test coverage is a semi automatic way of partitioning inputs domain
dato dalla percentuale di codice eseguita con un set di test 

$100\%$ coverage non significa bug free, non trovo i bug di omissione ad esempio

branch coverage per gli if (se testo sia il caso true che false)
loop coverage testo con 0 iterazioni, 1 iterazione, 1+ iterazioni 

modified condition decision coveragee MC/DC 
    branch coverage + tutte le possibili combinazioni se ho pi\'u if

path coverage, testa come finisco in un certo pezzo di codice.

\section{API design}
Use limit and offset to allow pagination	
–  /movies?limit=20&offset=0	
  