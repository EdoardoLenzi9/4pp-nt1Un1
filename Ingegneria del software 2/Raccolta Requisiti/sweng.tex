\chapter{Contesto}

App per visualizzare i ritardi degli autobus.

\section{Dati}
Trentino Trasporti non fornisce i dati, dunque si pu\'o utilizzare il \textbf{crowdsourcing}.

\section{Prodotto}
L’app avr\'a una sezione dove \'e possibile consultare i dati disponibili sui 
ritardi e avr\'a una sezione per l’inserimento.
La sezione di consultazione conterr\'a dei banner pubblicitari per raggiungere due obiettivi:
\begin{itemize}
	\item Guadagno\\
	\item Retention\\
\end{itemize}
Infatti cosa dovrebbe spingere il mio utente a fare la fatica di inserire i dati per altre persone?
Ogni inserimento viene pagato con una \textbf{moneta interna all’app}, con un certo ammontare di monete 
\'e possibile pagare la rimozione della pubblicit\'a.
La rimozione di pubblicit\'a \'e un prodotto ad abbonamento mensile, pagabile sia con soldi veri che 
con la moneta interna dell’app. La possibilit\'a di pagare il servizio offerto con il proprio 
contributo \'e ci che dovrebbe creare retention.

\chapter{Funzionalit\'a}
\begin{itemize}
	\item login/registrazione per autenticare l'utente\\
	\item servizio di geolocalizzazione dell'utente\\
		\item [-] fatto il login sarebbe figo che il sistema capisse a quali fermate sono vicino e mi dice 
		i bus che passano nei prossimi minuti\\
	\item se vedo che il bus \'e in ritardo lo segnalo\\
	\item se pi\'u utenti segnalano correttamente guadagnano monete\\
	
