teleconsole

docker server e client

docker images | grep nginx 	
lista immagini salvare

immagine iso
container istanza dell'immagine

posso far partire in contemporanea due versioni della mia app senza farle entrare in conflitto

docker run -d --name nginx-1-12-2 nginx:1.12.2 
per far partire un immagine (in un container) 
-d per mandare in bg il container
--name il nome 
poi l'immagine:versione
mi dal l'hash che id il container 

docker ps 
docker ps -a
vedo la lista container viva sulla macchina 

docker exec -ti <id container> bash
per lanciare comandi dentro il container da terminale
exit per uscire

docker start <hash/nome>
docker stop <id/hash>

docker rm -f nginx-1.12.2
se in esecuzione rimuove e stoppa

il container espone una porta 80, del container non dell'host
posso fare binding fra le porte dell'host e del guest 

docker run ... -p 8080:80 mi binda la mia 8080 con la 80 del docker

docker exec -it nginx-1-12-7 bash
posso usare apt-get install 

\section{Passare file dal sistema al container}
Monto un volume dentro il container.
I container una volta istanziati sono immodificabili.

ogni container ha un suo ip, unico ed univoco
docker run ... -v "path_locale:path_docker"

se uccido il container il volume resta

docker run --name php -d php:fpm 

\section{network}
Mettere in comunicazione i container con una rete

docker network create webinar
docker network ls

docker run --name php -v 'pwd'/code
docker ls code 

docker run --name php -v 'pwd'/code:/code --newtwork=webinar -d php:fpm
docker ps
docker exec -it <hash> bash

posso metterne n di -v options

\section{docker file}
file di istruzioni per creare un immagine
Dockerfile

FROM nginx:1.13.7

RUN apt-get update && apt-get install -y vim
RUN ...
WORKDIR <path>
COMMAND istruzione che viene eseguita appena parte il container, faccio partire il db
ENTRYPOINT creo il db
#commenti
COPY src /dest

docker build -t webinar_nginx:0.1 .
. per la path corrente

così casha i comandi run, li esegue una sola volta

\section{Docker compose}
Docker compose per velocizzare setup

docker network remove webinar

docker-compose.yml
version: '2'

services:
	php: 
		image: php:fpm
		volumes: 
			- /relative/path:dest/path
			- /path2:/path2
	nginx:
		image: nginx:1.13.7
		volumes:
			- ...
		ports:
			-"8080:80"

traduco i comandi in un manifesto	
docker-compose up -d
docker-compose down
docker-compose ps
docker-compose -f <nome dockercompose.yml> up -d

slega dal so
reverse proxy

con l'opzione no-cache di docker build 

